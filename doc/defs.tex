\newcommand{\xsubsection}[1]{%
\texonly{\subsection{#1}}%
\htmlonly{\strong{#1}\\}%
}

\newcommand{\evalsto}{%
\texonly{$\rightarrow$}%
\htmlonly{\code{->}}%
}

\newcommand{\cvar}[1]{%
\texonly{{\rm\em{#1}}}%
\htmlonly{\code{\var{#1}}}%
}

%%%%%%%%%%%%%%%% Latex prototypes
\texonly{

% \newenvironment{ex}%
%   {\scriptsize\setlength{\parskip}{0pt}
%    \verbatim}
%   {\endverbatim\vspace*{-0.1in}}

\newenvironment{ex}{\small\begin{example}}{\end{example}}
\newenvironment{exv}{\scriptsize\verbatim}{\endverbatim}

% \newenvironment{exv}%
%   {\scriptsize
%    \begin{center}
%    \begin{Sbox}
%    \begin{minipage}{3.25in}
%    \verbatim}
%   {\endverbatim
%    \end{minipage}
%    \end{Sbox}
%    \end{center}
%    \shadowbox{\TheSbox}}

\newenvironment{packeditemize}
	{\begin{list}{$\bullet$}{
	  \setlength{\topsep}{0pt}
	  \setlength{\parsep}{0pt}
	  \setlength{\parskip}{0pt}
	  \setlength{\itemsep}{0pt}}}
	{\end{list}}

\newenvironment{defs}{}{}
%\newcommand{\kode}[1]{\scriptsize\verbatim#1\endverbatim}
\newcommand{\kode}[1]{{\scriptsize\tt #1}}
\newlength{\parenlength}
  \settowidth{\parenlength}{\kode{(}} % -factor
  
\newcommand{\snl}{\newline\hspace*{\parenlength}}

\setlength{\shadowsize}{1pt}

\newcommand{\kind}[1]{\hspace*{0.015in}\hfill\small{$\mathcal #1$}\hfill}

%\newlength{\defnamelength}\setlength{\defnamelength}{1.25in}
\newlength{\defnamelength}
  \settowidth{\defnamelength}{\kode{insensitive-}} % -factor
\newlength{\defkindlength}
  \settowidth{\defkindlength}{\kind{M}}
  \addtolength{\defkindlength}{-0.125in} % fudge
\newlength{\defsiglength}\setlength{\defsiglength}{\columnwidth}
  \addtolength{\defsiglength}{-1\defnamelength}  
  \addtolength{\defsiglength}{-1\defkindlength}
%  \addtolength{\defsiglength}{-0.3in} % fudge
% \addtolength{\defsiglength}{-0.15in} % fudge
% \addtolength{\defsiglength}{-0.20in} % fudge
  \addtolength{\defsiglength}{-0.165in} % fudge
%\newlength{\defsiglength}\settowidth{\defsiglength}{\kode{(owner getter|<gen> setter|<gen> type init|<fun>)}}
% \newlength{\defsiglength}\setlength{\defsiglength}{3in}
\newlength{\defcontnamelength}
  \setlength{\defcontnamelength}{\defnamelength} 
  \addtolength{\defcontnamelength}{-0.01in} % fudge

\typeout{textwidth  = \the\textwidth}
\typeout{columnwidth  = \the\columnwidth}
\typeout{namelength = \the\defnamelength}
\typeout{kindlength = \the\defkindlength}
\typeout{siglength  = \the\defsiglength}

\parindent 0pt

\newcommand{\dbox}{\shadowbox} %\fbox

\newcommand{\defnoindex}[3]{
  \dbox{
  \parbox[t]{\defnamelength}{\setlength{\baselineskip}{0.5\baselineskip}\kode{#1}}
  }
  \dbox{
  \parbox[t]{\defsiglength}{\kode{#2}}
  }
  \hfill\penalty 0
  \dbox{
  \makebox[\defkindlength][c]{\kind{#3}}
  }
  \newline}

\newcommand{\defcontnoindex}[2]{
  {\setlength{\fboxrule}{0pt}
  \framebox{
  \parbox{\defcontnamelength}{\setlength{\baselineskip}{0.5\baselineskip}\kode{\hfill}}
  }}
  \rule{0.1pt}{0mm}
  \dbox{
  \parbox[t]{\defsiglength}{\kode{#1}}
  }
  \hfill\penalty 0
  \dbox{
  \makebox[\defkindlength][c]{\kind{#2}}
  }
  \newline}

\newlength{\defdocindentlength}\setlength{\defdocindentlength}{0.25\defnamelength}
\newlength{\defdoclength}\setlength{\defdoclength}{0in}
  \addtolength{\defdoclength}{\defnamelength}
  \addtolength{\defdoclength}{\defsiglength}
  \addtolength{\defdoclength}{\defkindlength}
  \addtolength{\defdoclength}{2\defdocindentlength} % column mode
%  \addtolength{\defdoclength}{-1\defdocindentlength}
  \addtolength{\defdoclength}{0.0275in}

\newcommand{\doc}[1]{\noindent\makebox[\defdocindentlength]{}\fbox{\parbox{\defdoclength}{\small\it #1}}\newline}

% \newcommand{\doc}[1]{\noindent\makebox[\defdocindentlength]{}\parbox{\defdoclength}{\small\it #1}\newline}

%\hbox{\spaceskip=0.5em{{\rm #1}}}}

\newcommand{\define}[3]{\defnoindex{{#1}}{{#2}}{{#3}}} % \mainschindex{{#1}}}}
% \newcommand{\def}[3]{\defnoindex{{#1}}{{#2}}{{#3}}\mainschindex{{#1}}}

\newcommand{\subsubsubsection}[1]{\subsubsection*{#1}}

\newcommand{\form}[2]{\define{{#1}}{{#2}}{S}}
\newcommand{\formcont}[1]{\defcontnoindex{{#1}}{S}}
\newcommand{\syntax}[2]{\define{{#1}}{$\equiv$ {#2}}{L}}
\newcommand{\notation}[2]{\define{{#1}}{{\it #2}}{N}}
\newcommand{\alias}[2]{\define{{#1}}{{#2}}{A}}
\newcommand{\generic}[2]{\define{{#1}}{({#2})}{G}}
\newcommand{\method}[2]{\define{{#1}}{({#2})}{M}}
\newcommand{\methodcont}[1]{\defcontnoindex{({#1})}{M}}
\newcommand{\function}[2]{\define{{#1}}{({#2})}{F}}
\newcommand{\class}[2]{\define{{#1}}{({#2})}{C}}
\newcommand{\property}[2]{\define{{#1}}{({#2})}{P}}
\newcommand{\propertyinit}[3]{\define{{#1}}{({#2})}{P}}
\newcommand{\instance}[2]{\define{{#1}}{{#2}}{I}}
\newcommand{\command}[2]{\define{{#1}}{{#2}}{K}}
}
%%%%%%%%%%%%%%%% end of Latex def definitions

%%%%%%%%%%%%%%%% HTML deftypes
\htmlonly{
\newenvironment{defs}{\begin{itemize}}{\end{itemize}}

% The following is for deftypes that have return types.
%    (foo int int) -> int

\newcommand{\defnoindex}[3]{%
%\cindex{\code{#1}}%
\item\noindent\code{({#1}{\var{#2}\/})~-->~{\var{#3}}}}

\newcommand{\def}[3]{%
\defnoindex{{#1}}{{#2}}{{#3}}%
\mainschindex{{#1}}}%

\newcommand{\defnoresultnoindex}[2]{%
%\cindex{\code{#1}}%
\item\noindent\code{({#1}{\var{#2}\/})}}

\newcommand{\defnoresult}[2]{%
\defnoresultnoindex{{#1}}{{#2}}%
\mainschindex{{#1}}}%

\newcommand{\pconstdef}[2]{%
\item\noindent\deftagstart\code{{#1}}\deftag{{#2}}}

% Variable deftype
\newcommand{\constdef}[2]{\pconstdef{{#1}}{{#2}}\mainschindex{{#1}}}

\newcommand{\constdefnoindex}[2]{\pconstdef{{#1}}{{#2}}}

\newcommand{\cdef}[1]{%
\item\noindent\code{{#1}}}

\newcommand{\cgcdef}[1]{%
\item\noindent\deftagstart\code{{#1}}\deftag{(may GC)}}

\newcommand{\syntaxdefnoresult}[2]{%
\item\noindent\deftagstart\code{({#1}{#2})}\deftag{syntax}}

\newcommand{\defresult}[1]{%
\\{}\noindent\code{\ \ \ \ -->~{\it{#1}}}}

\newcommand{\syntaxdef}[3]{%
\item\noindent\deftagstart\code{({#1}{#2})~-->~{\var{#3}}}\deftag{syntax}}

\newcommand{\deftagstart}{%
\begin{rawhtml}<table border=0 cellspacing=0 cellpadding=0 width=80%>
<tr> <td>\end{rawhtml}}

\newcommand{\deftag}[1]{%
\begin{rawhtml}</td> <td align=right>\end{rawhtml}%
{#1}%
\begin{rawhtml}</td></tr></table>\end{rawhtml}}

}
%%%%%%%%%%%%%%%% end of HTML def definitions
